\documentclass[12pt]{article}

\title{Assignment 2}
\author{Group 5 - Robert Krency, Ryan Miller, Zach Teixido}
\date{}

\usepackage{subfiles}

\usepackage{tikz}
%Tikz Library
\usetikzlibrary{angles, quotes, intersections}

%Notation
\usepackage{physics}
\usepackage{bm}

%AMS
\usepackage{amsmath}
\usepackage{amsfonts}
\usepackage{amssymb}
\usepackage{amsthm}

%Stuff
\usepackage{graphicx}
\usepackage{tabularx}
\usepackage{multicol}
\usepackage{algpseudocode}
\usepackage{algorithm}
\usepackage{enumitem}

\usepackage{setspace}
\onehalfspacing

\usepackage{xcolor}
\usepackage{listings}

\definecolor{mGreen}{rgb}{0,0.6,0}
\definecolor{mGray}{rgb}{0.5,0.5,0.5}
\definecolor{mPurple}{rgb}{0.58,0,0.82}
\definecolor{backgroundColour}{rgb}{0.95,0.95,0.92}

% Listings Style for C Lang
\lstdefinestyle{CStyle}{
    backgroundcolor=\color{backgroundColour},   
    commentstyle=\color{mGreen},
    keywordstyle=\color{magenta},
    numberstyle=\tiny\color{mGray},
    stringstyle=\color{mPurple},
    basicstyle=\footnotesize,
    breakatwhitespace=false,
    breaklines=true,                 
    captionpos=b,                    
    keepspaces=true,                 
    numbers=left,                    
    numbersep=5pt,                  
    showspaces=false,                
    showstringspaces=false,
    showtabs=false,                  
    tabsize=2,
    language=C
}

\lstdefinestyle{customasm}{
  belowcaptionskip=1\baselineskip,
  frame=L,
  xleftmargin=\parindent,
  language=[x86masm]Assembler,
  basicstyle=\footnotesize\ttfamily,
  commentstyle=\itshape\color{purple!40!black},
}

% Geometry 
\usepackage{geometry}
\geometry{letterpaper, left=1in, top=1in, right=1in, bottom=1in}

\setlength\columnsep{20pt}

% Fancy Header
\usepackage{fancyhdr, lastpage}
\fancypagestyle{plain}{
  \fancyhf{}    % Clear header/footer
  \fancyhead[L]{ CMSC 3240 - Computer Architecture}
  \fancyhead[R]{PennWest - F23}
  \fancyfoot[C]{Page \thepage\ of \pageref{LastPage}}
}
\pagestyle{plain}

% Add vertical spacing to tables
\renewcommand{\arraystretch}{1.8}

% Macros
\newcommand{\definition}[1]{\underline{\textbf{#1}}}

\newenvironment{rcases}
  {\left.\begin{aligned}}
  {\end{aligned}\right\rbrace}

% Begin Document
\begin{document}

\maketitle

\subsection*{Solution:}

\begin{lstlisting}[style=customasm]
  ; Initialize values
  loadA   0x00        ; Load 0 into the accumulator
  store   0xB0        ; Store the 0 into Sum, address 0xB0
  loadA   0x01        ; Load 1 into the accumulator
  store   0xB1        ; Store 1 into X, address 0xB1
  loadA   0x0A        ; Load 10 into the accumulator
  store   0xB2        ; Store 10 as the loop upper bound
  loadA   0x01        ; Load 1 into the Accumulator
  store   0xB3        ; Store the increment value

  ; loop test
  loadA   0xB1        ; Load the value of X into the accumulator
  comp    0xB2        ; Compare X < 10
  boz     0x12        ; If X == 10, jump to termination

  ; loop block
  loadA   0xB1        ; Load X into the accumulator
  add     0xB0        ; Add Sum+X
  store   0xB0        ; Store the Sum

  ; Increment X
  loadA   0xB1        ; Load X into the accumulator
  add     0xB3        ; Add 1 to X
  store   0xB1        ; Store X
  jump    0x08        ; Jump back to the Loop Test

  ; termination
  stop                ; terminate program
\end{lstlisting}


\end{document}